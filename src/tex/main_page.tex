%! Author = kyoto
%! Date = 11.09.2023

\vspace{3cm}
\tableofcontents

\newpage

\section{Задание}

По варианту, выданному преподавателем, составить и выполнить запросы к базе данных "Учебный процесс".\\
Команда для подключения к базе данных ucheb:\\
\begin{verbatim}
	psql -h pg -d ucheb
\end{verbatim}

\newpage
\section{Сделать запрос для получения атрибутов из указанных таблиц, применив фильтры по указанным условиям:}
Таблицы: \textit{Н\_ЛЮДИ, Н\_СЕССИЯ}. \\
Вывести атрибуты: \textit{Н\_ЛЮДИ.ОТЧЕСТВО, Н\_СЕССИЯ.ЧЛВК\_ИД}. \\
Фильтры (\textit{AND}): \\
\begin{enumerate}
	\item \textit{Н\_ЛЮДИ.ИМЯ} < \textbf{Николай}
	\item \textit{Н\_СЕССИЯ.ЧЛВК\_ИД} > \textbf{105948}
	\item \textit{Н\_СЕССИЯ.ЧЛВК\_ИД} < \textbf{126631}
\end{enumerate}
Вид соединения: \textit{INNER JOIN}.

\subsection{SQL-Запрос:}
\begin{verbatim}
select Н_ЛЮДИ.ОТЧЕСТВО, Н_СЕССИЯ.ЧЛВК_ИД
from Н_ЛЮДИ
	join Н_СЕССИЯ on Н_ЛЮДИ.ИД = Н_СЕССИЯ.ЧЛВК_ИД
	where Н_ЛЮДИ.ИМЯ < 'Николай'
		and Н_СЕССИЯ.ЧЛВК_ИД > 105948
		and Н_СЕССИЯ.ЧЛВК_ИД < 126631;
\end{verbatim}

\newpage
\section{Сделать запрос для получения атрибутов из указанных таблиц, применив фильтры по указанным условиям:}
Таблицы: \textit{Н\_ЛЮДИ, Н\_ВЕДОМОСТИ, Н\_СЕССИЯ}. \\
Вывести атрибуты: \textit{Н\_ЛЮДИ.ИД, Н\_ВЕДОМОСТИ.ИД, Н\_СЕССИЯ.ДАТА}. \\
Фильтры (\textit{AND}): \\
\begin{enumerate}
	\item \textit{Н\_ЛЮДИ.ИД} < \textbf{163484}.
	\item \textit{Н_ВЕДОМОСТИ.ИД} < \textbf{1490007}
\end{enumerate}
Вид соединения: \textit{RIGHT JOIN}.

\subsection{SQL-Запрос:}
\begin{verbatim}
select Н_ЛЮДИ.ИД, Н_ВЕДОМОСТИ.ИД, Н_СЕССИЯ.ДАТА
from Н_ЛЮДИ
	right join Н_ВЕДОМОСТИ on Н_ЛЮДИ.ИД = Н_ВЕДОМОСТИ.ЧЛВК_ИД
	right join Н_СЕССИЯ using(ЧЛВК_ИД)
	where Н_ЛЮДИ.ИД < 163484
	and Н_ВЕДОМОСТИ.ИД < 1490007;
\end{verbatim}

\newpage
\section{Вывести число студентов группы 3102, которые старше 25 лет.}
Ответ должен содержать только одно число.

\subsection{SQL-Запрос:}
\begin{verbatim}
select count(*)
from Н_УЧЕНИКИ
	join Н_ЛЮДИ on ЧЛВК_ИД = Н_ЛЮДИ.ИД
	where ГРУППА = '3102'
	and ДАТА_РОЖДЕНИЯ < CURRENT_DATE - interval '25 years';
\end{verbatim}

\newpage
\section{Выдать различные фамилии студентов и число людей с каждой из этих фамилий, ограничив список фамилиями, встречающимися менее 10 раз на кафедре вычислительной техники.}
Для реализации использовать подзапрос.
\subsection{SQL-Запрос:}
\begin{verbatim}
select ФАМИЛИЯ, count(*) as cnt
from Н_ЛЮДИ
    where ФАМИЛИЯ in (
        select ФАМИЛИЯ
        from Н_ЛЮДИ
            join Н_УЧЕНИКИ on Н_ЛЮДИ.ИД = Н_УЧЕНИКИ.ЧЛВК_ИД
            join Н_ПЛАНЫ on Н_УЧЕНИКИ.ПЛАН_ИД = Н_ПЛАНЫ.ИД
            join Н_ОТДЕЛЫ using(ОТД_ИД)
            where Н_ОТДЕЛЫ.КОРОТКОЕ_ИМЯ = 'ВТ'
            group by ФАМИЛИЯ
            having count(*) < 10
        )
    group by ФАМИЛИЯ;
\end{verbatim}

\newpage
\section{Выведите таблицу со средними оценками студентов группы 4100 (Номер, ФИО, Ср\_оценка), у которых средняя оценка не меньше минимальной оценк(е|и) в группе 1100.}
\subsection{SQL-Запрос:}
\begin{verbatim}
select ГРУППА, ФАМИЛИЯ, ИМЯ, ОТЧЕСТВО, avg(СОРТ)
from Н_УЧЕНИКИ
    join Н_ЛЮДИ on Н_ЛЮДИ.ИД = Н_УЧЕНИКИ.ЧЛВК_ИД
    join Н_ВЕДОМОСТИ using(ЧЛВК_ИД)
    join  Н_ОЦЕНКИ on ОЦЕНКА=КОД
    where ГРУППА='4100'
    group by ГРУППА, ФАМИЛИЯ, ИМЯ, ОТЧЕСТВО
    having avg(СОРТ) <= (
            select MAX(СОРТ)
            from Н_УЧЕНИКИ
                join Н_ВЕДОМОСТИ using(ЧЛВК_ИД)
                join  Н_ОЦЕНКИ  on ОЦЕНКА=КОД
                where ГРУППА='1100'
        );
\end{verbatim}

\newpage
\section{Получить список студентов, отчисленных после первого сентября 2012 года с заочной формы обучения (специальность: 230101). В результат включить:}
\begin{enumerate}
	\item номер группы
	\item номер, фамилию, имя и отчество студента
	\item номер пункта приказа
\end{enumerate}

Для реализации использовать подзапрос с \textit{IN}.

\subsection{SQL-Запрос:}
\begin{verbatim}
select ГРУППА, ЧЛВК_ИД, ФАМИЛИЯ, ИМЯ, ОТЧЕСТВО, П_ПРКОК_ИД
from Н_УЧЕНИКИ
    join Н_ЛЮДИ on Н_УЧЕНИКИ.ЧЛВК_ИД = Н_ЛЮДИ.ИД
    join Н_ПЛАНЫ on Н_ПЛАНЫ.ИД = Н_УЧЕНИКИ.ПЛАН_ИД
    join Н_НАПРАВЛЕНИЯ_СПЕЦИАЛ on Н_ПЛАНЫ.НАПС_ИД = Н_НАПРАВЛЕНИЯ_СПЕЦИАЛ.ИД
    join Н_НАПР_СПЕЦ on Н_НАПР_СПЕЦ.ИД = Н_НАПРАВЛЕНИЯ_СПЕЦИАЛ.НС_ИД
    join Н_ФОРМЫ_ОБУЧЕНИЯ on Н_ФОРМЫ_ОБУЧЕНИЯ.ИД = Н_ПЛАНЫ.ФО_ИД
    where Н_НАПР_СПЕЦ.КОД_НАПРСПЕЦ = '230101'
        and Н_ФОРМЫ_ОБУЧЕНИЯ.НАИМЕНОВАНИЕ = 'Заочная'
        and Н_УЧЕНИКИ.ПРИЗНАК = 'отчисл'
        and Н_УЧЕНИКИ.КОГДА_ИЗМЕНИЛ > '2012-09-01'::date
\end{verbatim}

\newpage
\section{Вывести список студентов, имеющих одинаковые отчества, но не совпадающие даты рождения.}

\subsection{SQL-Запрос:}
\begin{verbatim}
select DISTINCT ON( Н_ЛЮДИ.ОТЧЕСТВО, Н_ЛЮДИ.ДАТА_РОЖДЕНИЯ) Н_ЛЮДИ.ФАМИЛИЯ, Н_ЛЮДИ.ИМЯ, Н_ЛЮДИ.ОТЧЕСТВО, Н_ЛЮДИ.ДАТА_РОЖДЕНИЯ
from Н_УЧЕНИКИ
    join Н_ЛЮДИ on Н_УЧЕНИКИ.ЧЛВК_ИД = Н_ЛЮДИ.ИД
    order by Н_ЛЮДИ.ОТЧЕСТВО
\end{verbatim}