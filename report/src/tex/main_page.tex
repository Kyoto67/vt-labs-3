%! Author = kyoto
%! Date = 25.09.2023

\tableofcontents

\newpage


\section{Введение:}
Требуется создать базу знаний в языке программирования Prolog и реализовать набор запросов, используя эту базу знаний. Задача направлена на развитие навыков работы с фактами, предикатами, и правилами в логическом программировании.
Целью этой лабораторной работы является знакомство со средой разработки онтологий Protege и перевод базы знаний, созданной в предыдущей лабораторной работе в онтологическую форму в Protege.
Целью этой лабораторной работы является разработка программы, которая будет использовать базу знаний или онтологию для предоставления рекомендаций на основе введенных пользователем данных. (Knowledge-based support system)


\section{Анализ требований:}
- Корректность базы знаний и выполненных запросов.
- Сложность и разнообразие запросов.
- Качество документации и комментариев к коду.
- Корректное создание онтологии в Protege на основе базы знаний в Prolog.
- Качество перевода фактов, предикатов и отношений из Prolog в онтологию.
- Определение классов, свойств и иерархии классов в Protege.
- Тестирование онтологии и демонстрация ее функциональности (визуализация и проверка запросов).
- Создать программу, которая позволяет пользователю ввести запрос через командную строку. Например, информацию о себе, своих интересах и предпочтениях в контексте выбора видеоигры - на основе фактов из БЗ (из первой лабы)/Онтологии(из второй).
- Использовать введенные пользователем данные, чтобы выполнить логические запросы к БЗ/Онтологии.
- На основе полученных результатов выполнения запросов, система должна предоставить рекомендации или советы, связанные с выбором из БЗ или онтологии.


\section{Реализация системы искусственного интеллекта на Prolog}
\begin{verbatim}
% Facts with one argument
is_ghoul("kaneki").
is_ghoul("touka").
is_ghoul("ayato").
is_human("amon").
is_human("akira").
is_investigator("furuta").
is_investigator("juzo").
is_investigator("haise").
is_investigator("mado").

is_ghoul_investigator("kaneki").
is_ghoul_investigator("haise").
is_ghoul_investigator("furuta").

% Facts with two arguments
is_friend("kaneki", "touka").
is_enemy("kaneki", "amon").
is_superior("furuta", "juzo").
is_subordinate("juzo", "furuta").
is_part_of("amon", "ccg").
is_part_of("kaneki", "ghouls").
is_part_of("akira", "ccg").
is_part_of("haise", "ccg").
is_part_of("haise", "quinx").

has_kagune("kaneki", "rinkaku").
has_kagune("touka", "ukaku").
has_kagune("ayato", "ukaku").
has_quinque("amon", "jason").
has_quinque("akira", "arata").
has_quinque("haise", "ukaku").

is_kagune_type("rinkaku", "sss").
is_kagune_type("ukaku", "ss").
is_kagune_type("bikaku", "s").
is_quinque_type("jason", "ss").
is_quinque_type("arata", "sss").
is_quinque_type("ukaku", "s").

% Relationships between facts
is_enemy(X, Y) :- is_ghoul(X), is_investigator(Y).
is_enemy(Y, X) :- is_ghoul(X), is_investigator(Y).

is_superior(X, Y) :- is_investigator(X), is_investigator(Y), X\=Y.
is_subordinate(Y, X) :- is_investigator(X), is_investigator(Y), X\=Y.
is_part_of(X, Y) :- is_ghoul(X), is_ghoul(Y).
is_part_of(X, Y) :- is_investigator(X), is_investigator(Y).

% Rules
is_ghoul_enemy(X, Y) :- is_ghoul(X), is_ghoul(Y), X\=Y, is_enemy(X, Y).
is_investigator_enemy(X, Y) :- is_investigator(X), is_investigator(Y), X\=Y, is_enemy(X, Y).
is_ghoul_friendly(X, Y) :- is_ghoul(X), is_ghoul(Y), X\=Y, is_friend(X, Y).
is_investigator_friendly(X, Y) :- is_investigator(X), is_investigator(Y), X\=Y, is_friend(X, Y).

\end{verbatim}


\section{Оценка и интерпретация результатов:}
?- is_ghoul("kaneki").
true.

?- is_ghoul("amon").
false.

?- is_part_of(X,"ccg").
X = "amon" ;
X = "akira" ;
X = "haise" ;
false.

?- is_subordinate("furuta", X).
X = "juzo" ;
X = "haise" ;
X = "mado".

?- is_investigator_friendly(X,Y).
false.


Who have kagune "ukaku"?
Found 2 ghouls with "ukaku" kagune
1. touka
2. ayato
>

\section{Заключение}
    Система искусственного интеллекта на базе Prolog с использованием баз знаний и онтологий представляет собой мощный инструмент для решения различных задач, связанных с логическим выводом, обработкой знаний и анализом данных. Вот некоторые из ее преимуществ и потенциальных применений:

Преимущества:

1. Логический вывод: Prolog предоставляет механизм для логического вывода, который позволяет системе рассуждать и принимать решения на основе логических правил и фактов. Это делает систему мощным инструментом для решения задач, связанных с логикой и интеллектуальным анализом.

2. Удобство представления знаний: Prolog позволяет удобно представлять знания с использованием правил и фактов. Это делает систему легко настраиваемой и расширяемой, позволяя добавлять новые знания и правила без необходимости переписывания всей системы.

3. Онтологии: Использование онтологий позволяет системе организовать знания в иерархическую структуру с определением отношений между понятиями. Это облегчает семантический анализ и обработку информации.

4. Поддержка интеграции данных: Prolog может быть использован для интеграции данных из различных источников, что полезно для создания единой базы знаний и анализа информации из разных областей.

5. Рекомендательные системы: Системы, основанные на Prolog, могут быть использованы для разработки рекомендательных систем, которые анализируют предпочтения пользователей и предлагают им наилучшие варианты.

6. Экспертные системы: Prolog часто применяется для создания экспертных систем, которые способны давать советы и принимать решения в специфических областях, таких как медицина, финансы и техническая поддержка.

7. Анализ данных: Системы на базе Prolog могут использоваться для анализа структурированных данных, включая биоинформатику, анализ текста и многие другие области.

Потенциальные применения:

1. Медицинская диагностика: Создание экспертных систем, которые могут помогать врачам в диагностике заболеваний и выборе лечения.

2. Управление знаниями в предприятии: Применение системы для организации и поиска знаний в корпоративной среде.

3. Автоматизированный анализ текста: Использование системы для анализа текстовых данных, включая обработку естественного языка и извлечение информации.

4. Семантический веб: Создание семантического веба, где информация организована с использованием онтологий и может быть легко связана и анализирована.

5. Поддержка принятия решений: Применение системы для анализа данных и выдачи рекомендаций в областях финансов, логистики и управления ресурсами.

6. Обработка естественного языка: Разработка приложений для анализа и понимания человеческой речи, таких как чат-боты и системы автоматического ответа на вопросы.

7. Робототехника: Применение системы для управления и программирования роботов с целью автономной навигации и выполнения задач.

Системы искусственного интеллекта на базе Prolog с базами знаний и онтологиями обладают большим потенциалом для решения разнообразных задач и предоставления ценных решений в различных областях. Их гибкость и способность к логическому выводу делают их мощным инструментом для обработки знаний и данных.